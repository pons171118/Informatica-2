\documentclass[10pt,a4paper]{article}
\usepackage[utf8]{inputenc}
\usepackage{amsmath}
\usepackage{amsfonts}
\usepackage{amssymb}
\usepackage[left=3cm,right=3cm]{geometry}
\begin{document}


\begin{center}
\huge {HT4 Genericos} \\
\large {Mario Fernando Pons Fajardo} \\
\large {23 de febrero 2018}

\end{center}

\section*{Ventajas de los genericos}
Existen varias ventajas que tienen los métodos genéricos sobre otros métodos como los que reciben como parámetro un objeto. Las ventajas son las siguientes:

\begin{enumerate}
			\item{La carga seguridad la transfiere al compilador.  }
			\item{No existe la necesidad de escribir codigo para comprobar que el dato es correcto ya que se establece durante la compilación}
			\item{Heredar a un tipo base y cambiar miembros no es necesario.}

			\item{Mejora el rendimiento para guardar y controlar tipos de valor ya que no hay necesidad de que existan conversiones de datos.}

			\item{Habilitan devoluciones de llamada con seguridad}


\end{enumerate} 
 
\end{document}