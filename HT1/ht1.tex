\documentclass[10pt,a4paper]{article}
\usepackage[utf8]{inputenc}
\usepackage{amsmath}
\usepackage{amsfonts}
\usepackage{amssymb}
\usepackage[left=3cm,right=3cm]{geometry}

\begin{document}



\begin{center}
\huge {Hoja de trabajo no.1} \\
\large {Mario Fernando Pons Fajardo} \\
\large {25 de enero 2018}

\end{center}

\section*{A. Lista de "Que hacer"}

\begin{enumerate}
			\item{string Nombre de la actividad (El nombre se usara para poder identificar de manera eficaz la actividad a realizar.)}
			\item{string Fecha/hora (Se especificara la fecha que se tiene que realizar la actividad para que el responsable pueda organizarse para ejecutar la actividad,y la hora precisa.}
			\item{string Individuo/s que realizaran la actividad (Se identificara el responsable del "Que hacer" }

			\item{string Lugar (Se identificara un lugar para que el responsable o los responsables de la actividad sepan en donde tienen que realizarla.)}

			\item{string Herramientas (Que cosas necesita para realizar la tarea.)}


\end{enumerate}


\section*{B. Lista de "Que haceres"}

\begin{enumerate}
        \item{ Buscar: se podran buscar las tareas para que sean realizables de mejor manera.}
        \item{ Modificar:Para poder editar las tareas. }
        \item{ Agregar: Poder agregar nuevas tareas. }
        \item{ Eliminar:Poder borrar una tarea que querramos quitar. }
        \item{ Guardar: Guardar los cambios que se hayan efectuado.}
\end{enumerate}



\end{document}
